\documentclass[]{article}
\usepackage{amsthm,amsfonts}
\usepackage{mathtools}
\usepackage[brazilian]{babel}
\usepackage[utf8]{inputenc}
\usepackage[T1]{fontenc}

\begin{document}

\section{Recorrência}
Uma recorrência é uma expressão que dá o valor de uma função em termos dos valores "anteriores" da mesma função.

Para analisar o consumo de tempo de um algoritmo recursivo é necessário resolver uma recorrência.

O exemplo clássico de recorrência, provavelmente o mais fomoso, é a formula de Fibonacci:

$$
F(n) = \left\{
\begin{array}{ccc}
	0 &\mbox{se}\ n = 0\quad \\
	1 &\mbox{se n = 1}\quad \\
	F(n - 1) + F(n - 2) &\mbox{se} \ n \ge 2\quad
\end{array}\right.
$$

Outro exemplo clássico de recorrência é:
$$
F(n) = \left\{
\begin{array}{cc}
1 &\mbox{se n = 0}\quad \\
n.F(n - 1) &\mbox{se} \ n \ge 1\quad
\end{array}\right.
$$

Um fórmula fechada para F(n) é dada por:
$$F(n) = n.F(n - 1)$$
$$\quad = n.(n - 1).F(n - 2) = ... =$$
$$\quad = n.(n - 1).(n - 2).... 2.1.1$$
$$\quad = n!$$

Resolver uma recorrência é encontrar uma "fórmula fechada" que dê o valor da função diretamente em termos do seu argumento.
\\
Vamos analisar o consumo de tempo do algoritmo da busca binária.
\\
Em cada iteração, o algoritmo descarta metade do valor. Se denotamos por $T(n)$ o número máximo de iterações realizadas pela busca binária sobre um vetor com $n$ elementos, a função $T(n)$ pode ser expressa pela seguinte recorrência:
$$
T(n) = T(\lfloor \frac{n}{2} \rfloor) + 1, \ T(1) = 1.
$$

Para obter uma solução, supomos inicialmente que $n = 2^k$.


\begin{equation}
\begin{split}
T(n) = T(\lfloor \frac{n}{2} \rfloor) + 1 \\
= (T(\frac{n}{4}) + 1) + 1 \ = T(\frac{n}{2^2}) = 2\\
= T(\frac{n}{2^3}) + 3 \\
= T(\frac{n}{2^k}) + k = T(1) + k = k + 1 \\
= log \ {n} + 1
\end{split}
\end{equation}

Podemos agora tentar mostrar que $T(n) \le \log{n} + 1, \ \forall{n} \ge 1$, por indução em n.
\\
Para $n = 1, T(1) = 1 = \log{1} + 1$
Para $n > 1$, temos
\\\\\\\\\
continua...

\section{Recorrência - Exercicios}

\begin{enumerate}
		
		\item Resolva a recorrência $T(n)= T(\frac{n}{2}) + 7n + 2$, $T(1) = 1.$		

		\item Mostre que a solução de $T(n)= T(n - 1) + n$  é $O(n^2).$		

		\item Mostre que a solução de $T(n)= T(\lceil \frac{n}{2} \rceil{}) + 1$  é $O(\log n).$		

		\item Mostre que a solução de $T(n)= 2T(\lfloor \frac{n}{2} \rfloor{} + 17) + n$  é $O(n \log n).$		
	
		\item Resolva a recorrência $T(n)= 3T(\lfloor \frac{n}{2} \rfloor{}) + n$, $T(1) = 1.$		

		\item Resolva a recorrência $T(n)= 3T(\lfloor \frac{n}{3} \rfloor{}) + 1$, $T(1) = 1.$		
		
		\item Mostre que a solução de $T(n)= 3T(\frac{n}{2}) + n$  é $O(n^{\log_{2}{3}}).$	
		
		\item Mostre que a solução de $T(n)= 4T(\frac{n}{2}) + n$  é $O(n^2).$	
		
\end{enumerate}

\section{Respostas}

\begin{enumerate}
	\item  $T(n)= T(\frac{n}{2}) + 7n + 2$, $T(1) = 1.$ \\
	$T(n) \le O(n^2), \forall n \ge 8$, por indução em $n$
	\\
	
	$T(1) = (1)^2 = 1$ \\
	$T(2) = T(\frac{2}{2}) + 7*2 + 2 = 17 \not\le n^2(4) $ \\
	$T(4) = T(\frac{4}{2}) + 7*4 + 2 = 32 \not\le n^2(16) $ \\
	$T(8) = T(\frac{8}{2}) + 7*8 + 2 = 62 \le n^2(64) $ \\
	
	
\end{enumerate}

\end{document}